\chapter{Curvilinear Poisson FEM}
\label{chap:curvilinear_poisson_fem}

\section{Governing equations and functionals}

Start with the Poisson equation in weak form for a single element.  Solve for $u$; the following holds for all test functions $v$.
%
\begin{equation}
-\int dx\, dy\, \nabla v \cdot \nabla u = \int dx\, dy\, vf - \oint ds\, v e_n.
\end{equation}
%
Discretized using the matrices from the previous chapter, 
%
\begin{equation}
\boxed{
-\ve{v}^T \left[ \f{D_x^T} \q{Q} \f{D_x^T} + (\f{D_y^T} \q{Q} \f{D_y^T})\right] \ve u
=
\ve{v}^T \q{Q} \ve{f} - \sum_\textrm{edges} \ve{v}^T \q{Q}_\textrm{edge} \ve{e}_n.
}%boxed
\end{equation}

There are two objective functional types of particular interest to me: pointwise evaluation and integrals.  Pointwise evaluation:
%
\begin{equation}
\boxed{
f(u(x,y)) = f(I_f(R(x,y), S(x,y)) \ve u).
}%boxed
\end{equation}
%
Integration of $f(u)$ will be more accurate if $f$ is evaluated on quadrature points.  Throw in a weighting function $a(x,y)$ too.
%
\begin{equation}
\boxed{
\int dx \, dy \, a(x,y) f(u(x,y)) = a(X(\ve r, \ve s), Y(\ve r, \ve s))^T I_{f \rightarrow q}^T \q{V^{-T}} \q{V^{-1}} \operatorname{diag}(|\q{\ve J}|) f(I_{f \rightarrow q} \ve u).
}%boxed
\end{equation}
%
We must be careful to use $X$, $Y$, $R$ and $S$ appropriately or else these expressions will not work as expected in sensitivity analysis.

\subsection{Visualization}

Function visualization takes two steps: identify the element enclosing each output point, then interpolate the field within that element.  Let's assume the element has been correctly identified.  Then the evaluation at $\ve x, \ve y$ is
%
\begin{equation}
\boxed{
u(\ve x, \ve y) = I_f(R(\ve x, \ve y), S(\ve x, \ve y)) \ve u.
}%boxed
\end{equation}
%
Other functions $f(x,y)$ can be evaluated analytically.

\section{Sensitivity by direct differentiation}

\subsection{Stiffness matrix}

Perturb term by term.
%
\begin{equation}
\partial_p \left[ \f{D_x^T} \q{Q} \f{D_x^T} \right] \ve u = \left[ \partial_p \f{D_x^T} \q{Q} \f{D_x^T} + \f{D_x^T} \partial_p \q{Q} \f{D_x} + \f{D_x^T} \q{Q} \partial_p \f{D_x} \right] \ve u + \left[ \f{D_x^T} \q{Q} \f{D_x^T} \right] \partial_p \ve u.
\end{equation}
%
Unpack $\partial_p \q{Q}$:
%
\begin{equation}
\partial_p \q{Q} = I_{f \rightarrow q}^T \q{V^{-T}} \q{V^{-1}} \operatorname{diag}( \partial_p |\q{\ve J}| ) I_{f \rightarrow q}.
\end{equation}
%
The sensitivity of $\f{D_x}$ was derived before (Equation \ref{eqn:gradient_matrix_sensitivity}) and depends on the sensitivity of $\f{J}$.  The sensitivity of $\q{Q}$ depends on the sensitivity of $\q{J}$.

\subsection{Inhomogeneous term}

\begin{equation}
\begin{aligned}
\partial_p \left[ \q{Q} f(\ve x, \ve y) \right] =
(\partial_p \q{Q}) f(\ve x, \ve y)
+ \q{Q} \,
\left[
\partial_x f(\ve x, \ve y)^T \partial_p X(\ve r, \ve s)
+ \partial_y f(\ve x, \ve y)^T \partial_p Y(\ve r, \ve s)
+ \partial_p f(\ve x, \ve y)
\right]
\end{aligned}
\end{equation}
%
Take the gradient of $f(\ve x, \ve y)$ using the gradient matrices.  (Why not?  It's assumed to be a polynomial.)
%
\begin{equation}
\partial_p \left[ \q{Q} f(\ve x, \ve y) \right] =
(\partial_p \q{Q}) f(\ve x, \ve y)
+ \q{Q} \,
\left[
(\f{D_x} \ve f)^T \partial_p X(\ve r, \ve s)
+ (\f{D_y} \ve f)^T \partial_p Y(\ve r, \ve s)
+ \partial_p f(\ve x, \ve y)
\right].
\end{equation}
%
I think I'm screwing up some tensor index thing here but I'm tired.  Check it later.
















